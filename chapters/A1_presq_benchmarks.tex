The repository \texttt{MatchBox}\footnote{\url{https://github.com/ayllon/MatchBox/}}
contains the Python implementation of \PresQ and \Find used as a reference for the
performance comparison shown in chapter~\ref{chapter:presq}, together with instructions
to reproduce the results.

For facilitating the replicability of the results, the repository contains an \linebreak
\texttt{environment.yml} file that allows re-creating a working Python environment
with all the requirements installed using \href{https://docs.conda.io/en/latest/}{\texttt{Conda}}:

\begin{minted}{bash}
conda env create
conda activate matchbox
\end{minted}

\section{Bench-marking quasi-clique finding}

\begin{code}
\label{script:quasi_missing}
\caption{Benchmark quasi-clique search with a set of missing ratios}
\begin{minted}{bash}
for alpha in 0.05 0.10 0.15 0.20 0.25 0.30; do
    ./bin/benchmark-quasiclique.py --out results/quasi3.csv \
       --rank 3 \
       --cardinality 10 20 30 \
       --additional 0.5 \
       --repeat 15 \
       --missing-edges $alpha \
       --extra-edges 0 \
       --timeout 300
done
\end{minted}
\end{code}

\begin{code}
\label{script:quasi_spurious}
\caption{Benchmark quasi-clique search with a set of additional edges}
\begin{minted}{bash}
for beta in 0.2 0.4 0.6 0.8; do
    ./bin/benchmark-quasiclique.py --out results/quasi3.csv \
        --rank 3 \
        --cardinality 10 20 30 \
        --additional 0.5 \
        --repeat 15 \
        --missing-edges 0.1 \
        --extra-edges $beta \
        --timeout 1200
done
\end{minted}
\end{code}

\section{Bench-marking EDD finding}