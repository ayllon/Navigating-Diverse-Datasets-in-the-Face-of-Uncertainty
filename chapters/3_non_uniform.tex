As a result of the survey presented in chapter~\ref{chapter:literature_review},
we realized that most solutions treat files
separately, leaving it to the end-user to work out how they are related,
an observation shared by other authors~\cite{Silva2016}.
These files may have not yet been ingested into a database system and 
the schema may be unfamiliar and not adequately documented, or even be
composed of multiple files with heterogeneous schemes~\cite{alawini2016,zhang2015astronomy}.

In \emph{Data Mining in Astronomical Databases}, Borne
describes how data exploration of this kind of diverse datasets is, however,
relevant since it can drive to serendipitous discoveries~\cite{Borne2001}. He proposes
two groups of data mining in this respect:

\paragraph{Event based}

\begin{description}[leftmargin=1cm, labelindent=0.5cm]
  \item[Known events / known algorithms]
    Use physical models to locate known phenomena of interest spatially
    or temporally within a large database
  \item[Known events / unknown algorithms]
    Use pattern recognition and clustering to discover new relationships between known
    phenomena.
  \item[Unknown events / known algorithms]
    Use predictive models to predict the presence of unseen events within a
    large and complex database.
  \item[Unknown events / unknown algorithms]
    Use thresholds to identify transifent or unique events.
\end{description}

\paragraph{Relationship based}

\begin{description}[leftmargin=1cm, labelindent=0.5cm]
    \item[Spatial]
        Identify objects in the same location.
    \item[Temporal]
        Identify events occurring within the same time period.
    \item[Coincidence]
        In general, apply clustering techniques to
        identify objects that are co-located within a multidimensional space.
\end{description}

Borne then enumerates a list of science requirements for data mining:

\begin{description}[leftmargin=1cm, labelindent=0.5cm]
    \item[Object Cross-Identification] between catalogs. Similar to the natural join
        in relational algebra, except based on spatial or multidimensional co-location.
    \item[Object Cross-Correlation] comparing sets of attributes over the full set of objects.
        For instance, identify remote galaxies as those that are \emph{not} present on the ultraviolet.
    \item[Nearest-neighbor identification] or, in general, application of clustering algorithms in multidimensional spaces.
    \item[Systematic Data Exploration] via event- and relationship-based queries to a database
        hoping to make serendipitous discoveries.
\end{description}

\section{Examining real use cases}

To get a better idea of what kind of explorations astronomers do, I looked for
concrete examples of queries mentioning \emph{astronomy} on the 242 articles classified on
the systematic literature mapping from chapter~\ref{chapter:literature_review}.

It is soon evident that the datasets published by \gls{SDSS}~\cite{SDSS14} are popular as test
sets since they are readily available and well documented~\cite{Gray2002}.
Furthermore, there are sample queries available~\cite{SDSSSamples}, and real ones can also be
obtained~\cite{SDSSSqlLogs}. Listing~\ref{sql:get_queries} shows an example of how to obtain a list
of queries performed by users.

\begin{listing}[htbp]
\begin{minted}[linenos]{sql}
    SELECT clientIP, seq, statement, elapsed
    FROM SQLlog
    WHERE yy=2018 AND mm>=10 AND rows>0 AND dbname LIKE 'BestDR14%'
\end{minted}
\caption{Example of how to obtain a list of queries performed by users during the end of 2018 over the 14th data release}\label{sql:get_queries}
\end{listing}

In total, 25 articles -- a $10.3\%$ -- use \gls{SDSS} as a test dataset.
Table~\ref{tab:sdss_queries_count} classifies these 25 articles following the same schema as described
in section~\ref{sec:mapping_category}.

\begin{table}[htbp]
  \begin{center}
    \begin{tabular}{l r r r}
      \textbf{Category} & \textbf{Total} & \textbf{SDSS} & \textbf{\%} \\ \hline
      Exploration Interfaces & 34 & 3 & $8.8\%$ \\
      Indexes & 58 & 5 & $8.6\%$ \\
      Storage & 39 & 11 & $28.3\%$ \\
      Data Visualization & 36 & 2 & $5.6\%$ \\
      Interactive Performance Optimizations & 48 & 4 & $8.3\%$ \\
    \end{tabular}
  \end{center}
  \caption{Classification of the articles that use the data from the \gls{SDSS}}\label{tab:sdss_queries_count}
\end{table}

To better understand how this dataset is used, I processed the results from query~\ref{sql:get_queries}
and extracted the columns, relations, and filters.
Table~\ref{tab:most_tables} shows the most frequent queried combinations.

\begin{table}[htbp]
\centering
\begin{tabular}{l r r}
    \textbf{Tables} & \textbf{Count} & \textbf{Percentage} \\ \hline
    fGetNearbyObjEq, PhotoPrimary          &  264785 &  37.62\% \\
    DBObjects\textasteriskcentered         &  150458 &  21.37\% \\
    PhotoTag, fGetObjFromRectEq            &   93094 &  13.23\% \\
    IndexMap\textasteriskcentered          &   41265 &   5.86\% \\
    Galaxy, fGetNearbyObjEq                &   31130 &   4.42\% \\
    sppParams, PhotoTag, fGetObjFromRectEq &   29805 &   4.23\% \\
\end{tabular}
\caption{Combination of relations most frequently queries. Tables marked with (\textasteriskcentered) are meta-data tables (i.e., describe the schema)}\label{tab:most_tables}
\end{table}

Interestingly, introspection queries are widespread, indicating that users spend considerable
time familiarizing themselves with the schema.

\todo{Actualizar con el informe 2019 el capítulo 2}

\section{Insights}

\paragraph{Múltiples ficheros} Cuando se quieren realizar consultas sobre múltiples
ficheros, la estructura de datos puede no coincidir. Ejemplo simple, en un fichero el
identificador único para una galaxia puede llamarse \texttt{ID}, y en otro \texttt{OBJECT\_ID}.
Izquierdo+\cite{Izquierdo2013} se enfrentan con un problema muy similar: ¿cómo averiguar
qué atributos son los mismos? Quizá, en nuestro caso, se pueda emplear el tipo y la distribución de los valores
como una pista adicional: por ejemplo, difícilmente se correspondan un atributo de la tabla $A$ con
valores entre \numprint{0} y \numprint{6}, con un atributo de la tabla $B$ con valores entre \numprint{0}
y \numprint{1e8}. Sin embargo, un rango entre \numprint{0} y \numprint{1e8} (flujo) puede estar
relacionado con otro con valores comprendidos entre \numprint{-4.4} y \numprint{27.0} (magnitud).

\todo{Clustering enlaza con \ref{chapter:som}}.





Therefore, our goal is to assist users in understanding how multiple raw files are related;
identifying shared sets of attributes, and facilitating
relationship-based mining between different files with heterogeneous schemes.

In chapter~\ref{chapter:presq} we propose a novel algorithm, \PresQ, that fills this gap for
uncertain numerical data.