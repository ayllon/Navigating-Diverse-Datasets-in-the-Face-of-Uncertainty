\begin{otherlanguage}{spanish}
{\fontsize{11}{11}\selectfont
Uno de los mayores problemas del \textit{big data} es el origen diverso de los datos.
Un investigador puede estar interesado en agregar datos provenientes de múltiples
ficheros que aún no han sido pre-procesados e insertados en un sistema de bases de datos,
debiendo depurar y filtrar el contenido antes de poder extraer conocimiento.

La exploración directa de estos ficheros presentará serios problemas de rendimiento:
examinar archivos sin ningún tipo de preparación ni indexación puede ser ineficiente tanto
en términos de lectura de datos como de tiempo de ejecución. Por otro lado, ingerirlos en
un sistema de base de datos antes de entenderlos
introduce latencia y trabajo potencialmente redundante si el esquema
elegido no se ajusta a las consultas que se ejecutarán. Afortunadamente, nuestra revisión del estado del arte
demuestra que existen múltiples soluciones posibles para explorar datos \emph{in-situ} de manera efectiva.

Otra gran dificultad es la gestión de archivos de diversas procedencias, ya que su esquema y disposición pueden
no ser compatibles, o no estar correctamente documentados. La mayoría de las soluciones encontradas
pasan por alto esta problemática, especialmente en lo referente a datos numéricos e inciertos,
como, por ejemplo, aquellos relacionados con atributos físicos generados en campos como la astronomía.

Nuestro objetivo principal es ayudar a los investigadores a explorar este tipo de datos sin
procesamiento previo, almacenados en múltiples archivos, y empleando únicamente su distribución intrínseca.

En esta tesis primero introducimos el concepto de \textit{Equally-Distributed Dependencies (EDD)}
(Dependencias de Igualdad de Distribución), estableciendo las bases necesarias para ser capaz
de emparejar conjuntos de datos con esquemas diferentes, pero con atributos en común.
Luego, presentamos \PresQ, un nuevo algoritmo probabilístico de búsqueda de \textit{quasi-cliques} en hiper-grafos.
El enfoque estadístico de \PresQ permite proyectar el problema de búsqueda de EDD en el de búsqueda
de quasi-cliques.

Por último, proponemos una prueba estadística basada en \textit{Self-Organizing Maps (SOM)} (Mapa autoorganizado).
Este método puede superar, en términos de poder estadístico, otras técnicas
basadas en clasificadores, siendo en algunos casos comparable a métodos basados en \textit{kernels},
con la ventaja adicional de ser interpretable.

Tanto \PresQ como la prueba estadística basada en SOM pueden impulsar descubrimientos serendípicos.
}
\end{otherlanguage}
