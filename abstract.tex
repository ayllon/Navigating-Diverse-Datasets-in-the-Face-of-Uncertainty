% Follow thesis structure
{\fontsize{11}{11}\selectfont
% Intro (Research questions!)
When exploring big volumes of data, one of the challenging aspects
is their diversity of origin. Multiple files that have not yet been ingested into a
database system may contain information of interest to a researcher,
who must curate, understand and sieve their content before being able to 
extract \emph{knowledge}.

% Review + Gaps
Performance is one of the greatest difficulties for \emph{exploring} these datasets.
On the one hand, examining non-indexed, unprocessed files can be inefficient. On the
other hand, any processing before its \emph{understanding} introduces latency and
potentially unnecessary work if the chosen schema matches poorly the data.
We have surveyed the state-of-the-art and, fortunately, there exist multiple
proposal of solutions to handle data \emph{in-situ} performantly.

% Objective
Another major difficulty is matching files from multiple origins since their
schema and layout may not be compatible or properly documented. Most surveyed solutions overlook this problem, especially for numeric, uncertain data, as is typical
in fields like astronomy.

The main objective of our research is to assist data scientists during the exploration of 
unprocessed, numerical, raw data distributed across multiple files based solely on its
intrinsic distribution.

In this thesis, we first introduce the concept of \emph{Equally-Distributed Dependencies}, which
provides the foundations to match this kind of dataset.
% PresQ
We propose \PresQ, a novel algorithm that finds quasi-cliques on hypergraphs based
on their expected statistical properties. The probabilistic approach of \PresQ can be
successfully exploited to mine EDD between diverse datasets when the underlying populations
can be assumed to be the same.

% SOM
Finally, we propose a two-sample statistical test based on Self-Organizing Maps,
a novel machine-learning-based technique for hypothesis testing. This method
can outperform, in terms of power, other classifier-based two-sample tests, being in some cases
comparable to kernel-based methods, with the advantage of being interpretable.

% Discussion + Conclusions (Results!)
Both \PresQ and the SOM-based statistical test can provide insights that drive serendipitous discoveries.
}
